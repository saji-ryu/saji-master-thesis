\chapter{結論}
\label{chap:conclusion}

本章では本研究を総括する。

\newpage



\section{研究の成果}
本研究では、NFCを利用したインタラクションとAR情報をwikiで管理するシステムを組み合わせた次世代のARナビゲーションシステム「HypAR Wiki」の提案を行った。

まず第\ref{chap:background}章において、ARによるナビゲーションの問題点を2次元上での既存メディアの進化と比較しながら分析した。ARでナビゲーションを行う既存システムの現状をとりあげ、ARナビゲーションシステムの問題点が根本的に解決されていないことを示した。

第\ref{chap:design}章では、第\ref{chap:background}章で述べたARナビゲーションシステムの問題点に対する有効的な解決方法を提案した。また、それに基づき本研究で開発した次世代ARナビゲーションシステム「HypAR Touch」の基本構成と使い方について述べた。

第\ref{chap:implementation}章では、「HypAR Touch」のアプリケーション構成と詳細な実装について述べた。

第\ref{chap:usage}章では、「HypAR Touch」によって実現可能な応用例について述べた。

第\ref{chap:relatedResearch}章では、本研究に関連する研究を紹介し、それぞれのアプローチの特徴と問題点を分析した。

第\ref{chap:consideration}章では、筆者による運用経験やユーザーからのフィードバックをもとに本研究の有効性と問題点を分析した。

\section{総括}
本研究ではNFCタグに触れるというインタラクションからARナビゲーションを表示でき、表示するAR情報の管理をwikiで行える「HypAR Touch」の開発を行った。
HypAR TouchはNFCタグに触れるインタラクションとWiki等の技術の組み合わせによって、ユーザーの位置推定の問題やAR情報の編集・参照が難しいといった問題を解決した。。
また有用な活用例を示し、既存のシステムより優れた点があることを示した。
今後は第\ref{chap:kosatsu}章で述べた問題点についての改善や、システム改善を行っていく。