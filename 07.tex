\chapter{考察}
\label{chap:consideration}

本章ではHypAR Touchの利用者の意見や自身の運用経験をまとめ、諸問題や新しい可能性について述べる。

\newpage


\section{評価}
本システムのうち、NFCタグによるインタラクション部分のプロトタイプとなる「TouchAR」は2019年後半から後半から開発を行っており、ORF2019\footnote{ \textsf{https://orf.sfc.keio.ac.jp/2019/} }にてDrawWikiの展示発表を行った。
またその後も2020年11月からの2ヶ月に渡り使用した様子を共有したり、実際にユーザーに使ってもらうことで意見を集めた。
本節ではユーザーからのフィードバックおよびTouchARの展示発表で得られた感想、筆者の運用経験をまとめる。

\subsection{意見・感想}

第\ref{mondai}節で述べたARによるナビゲーションの問題点や、その解決策としてNFC技術とwikiにより情報を管理する本システムについて多くの同意が得られた他、以下のような感想や意見が寄せられた。

\paragraph*{Wikiを採用したことによる編集の容易さ}
既存のAR表示システムでは情報の追加のフォーマットが決まっている事が多く、一般ユーザーが自発的に情報を追加編集することが難しい。
一方本システムではScrapboxページを作り、googleMapのURLを貼り付けるだけでARの情報を追加できるため、AR情報の追加という感覚なしに気軽に情報を追加できるという意見があった。
また既存のScrapboxのプロジェクトのうち地図情報があるものがそのままARで表示できる拡張性も評価された。

\paragraph*{NFCタッチによる起動}
既存のARナビゲーションシステムとして

\subsection{筆者の運用経験}
キャンパスでの利用を想定したフィールドワークを行った。
また自身の訪れたことのない場所の探索フィールドワークを複数回行った。

\paragraph*{リンクに基づく優れた参照性}
Scrapboxにはフォルダやタグ・ラベル等のメモやイラストを分類し管理するための機能は存在しない。
全てのAR情報がフラットに置かれているが、リンク情報に基づいて関連する情報が表示されるため



\subsection{問題点・要望}
以下のような問題点が明らかになった。
TODO:


\section{考察}

\subsection{設計の妥当性}
本システムは既存のものとは異なる新しい位置測位システムと情報管理を採用しているが実際に利用したりデモを体験したユーザーからは概ね好意的に受け入れられ、NFCタグを利用したインタラクションとwikiを利用した情報管理をARに組み合わせた本システムの設計指針は正しかったといえる。
また本研究で述べたARナビゲーションに対する問題意識にも多くの共感を得られたことから、本システムをベースにして様々な改善や拡張を行うことで、より良いARナビゲーションアプリを生み出せると考えられる。

\subsection{解決すべき課題}
TODO:

\subsection{問題点の検証}
本システムにおいて第2.3節で述べた問題点が克服されているかどうかを問題点毎に検証する。
\begin{itemize}
  \item 立ち上げるまでのインタラクションが面倒 \\
  NFCタグを利用することでタッチするだけでアプリを起動することができる。
  \item 位置測位の方法によって精度や用途が大きく限られる \\
  屋内/室外を問わずナビゲーションに十分な精度でARを表示できる
  \item 情報の登録・編集が面倒 \\
  wikiを利用し、ページとARで表示する情報を対応させることでで誰もが
  \item 関連情報を参照・管理することができていない \\
  ハイパーリンクを利用することで関連情報を参照・管理できる
  \item 汎用性のあるアプリケーションがない \\
  環境に左右されない位置測位とジャンルを問わずリンク参照から管理できる汎用性を持っていると言える。
\end{itemize}
以上のように、第2.3節で述べたすべての点に関して問題が解決していることがわかる。