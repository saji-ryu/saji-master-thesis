% ■ アブストラクトの出力 ■
%	◆書式:
%		begin{jabstract}〜end{jabstract}	:日本語のアブストラクト
%		begin{eabstract}〜end{eabstract}	:英語のアブストラクト
%		※ 不要ならばコマンドごと消せば出力されない。



% 日本語のアブストラクト
\begin{jabstract}
NFC技術をARの正確な位置測位とコンテキスト情報の取得に活かしつつ、AR情報の管理にWikiの手法を取り入れたARナビゲーションシステム、「HypAR Touch」を提案する。
モバイル端末によるARナビゲーションは近年普及し始めたが、(1)立ち上げるまでのインタラクションが面倒、(2)位置測位の方法によって精度や用途が大きく限られる、(3)情報の登録・編集が面倒、(4)関連情報を参照・管理することができていないなどといった問題が存在する。
HypAR TouchではNFC技術を利用することで正確な位置測位やコンテキスト情報の取得を可能とする。
さらに、AR情報の管理にWikiを採用することでハイパーリンクから関連する情報を簡単に参照管理することができる。
これによってARナビゲーションの問題点が解決されるだけでなく、リンクを使ったより探索的な使い方が可能になる。
本論文ではHypAR Touchの設計や実装、その応用例について述べ、研究の発展性について考察する。
\end{jabstract}



% 英語のアブストラクト
\begin{eabstract}
We propose \textit{HypAR Touch}, an AR navigation system that utilizes NFC technology for accurate positioning and contextual information acquisition in AR while incorporating Wiki for AR information management. 
AR navigation using mobile devices has been popular in recent years, but they have the following problems: 1. cumbersome interaction to start up, 2. accuracy and applications are greatly limited depending on the location positioning method, 3. registration and editing of information are not easy, and 4. reference and management of related information are not available. 
HypAR Touch uses NFC technology to enable accurate positioning and acquisition of context information. 
Furthermore, by adopting Wiki for managing AR information, related information can be easily referenced and handled through hyperlinks. 
HypAR Touch solves the problem of AR navigation and allows for a more exploratory use of links. 
We describe the design and implementation of HypAR Touch and its applications and discuss the research's development potential. 
\end{eabstract}
