\chapter{序論}
\label{chap:introduction}


本章では本研究の動機と目的、および本論文の構成について述べる。

\newpage


\section{研究の動機}
\label{motive}

拡張現実感(AR : Augmented Reality)によるヘルプ・ナビゲーションの歴史は長く、早いものでは1990年代から存在している。
またARにはヘッドマウントディスプレイを使うものと携帯端末のカメラを通した映像に情報を付加するものが存在するが、後者は近年のスマートフォンの普及と高性能化により利用環境が整って来ている。
しかし既存のARナビゲーションシステムには以下のような問題点があり、ARが汎用的なヘルプ・ナビゲーションシステムとして利用されていない現状がある。

\begin{itemize}
  \item 環境を問わず正確で安価に位置測位をすることが難しい
  \item 表示する情報の登録・編集が煩雑で参照や管理が面倒
  \item 案内を起動するまでの負荷が高い
\end{itemize}

一方でARでも頻繁に扱われるテキストや写真、地図などのマルチメディア情報は計算機の進歩とwebの発展とともに以下のような進化を遂げた。

\begin{itemize}
  \item 他の文書への参照を実現するハイパーリンクと、それを内包した文書であるハイパーテキストが登場した
  \item Webの普及によって様々なメディアにハイパーリンクを経由して手軽にアクセスできるようになった
  \item webからアクセス可能な地理情報システムが登場し地理情報の紐付けが用意になった
  \item コラボレーションツールであるWikiが複数人による共同編集を可能にし、知見の共有を実現した
\end{itemize}

さらにモバイル端末の高性能化により多くの端末で近距離無線通信(NFC : Near Field Communication)による非接触タグの読み書き機能が搭載されるようになっている。
NFCによる非接触タグには以下のような利点が存在する。

\begin{itemize}
  \item タグ側に電力を必要とせず、小型化できるためタグを設置する場所や物を選ばない
  \item 個別のIDやURL情報を記録するには十分な記憶容量を持つ
  \item 読み取り側で検知した時の動作をある程度規定できる
\end{itemize}

このような利点はヘルプシステムやナビゲーションシステムに利用するにあたって非常に有用なものであると考える。
\\
本研究ではNFCタグの利点をARの正確な位置測位とコンテキスト情報の取得に活かしつつ、AR情報の管理にWikiの手法を取り入れたシステムを開発し、既存のARナビゲーションシステムが抱える問題点を解決した。

\section{研究の目的}
本研究では、第\ref{motive}節で述べたARナビゲーションシステムが持つ問題点を解決するARナビゲーションシステム「HypAR Touch」の構築を目的とする。


\section{本論文の構成}

本論文は以下の8章で構成される。

第\ref{chap:background}章では、本研究の背景をより詳細に分析し、既存システムの問題点を整理する。

第\ref{chap:design}章では、本論文で提案するシステムの基本構成と使い方について述べる。

第\ref{chap:implementation}章では、本論文で提案するシステムの詳細な実装について述べる。

第\ref{chap:usage}章では、本論文で提案するシステムの利用例を紹介する。

第\ref{chap:relatedResearch}章では、関連する研究を紹介し、それらの特徴や本研究との関連を述べる。

第\ref{chap:consideration}章では、筆者による運用経験やユーザーからのフィードバックをまとめ、本論文で提案するシステムの有効性と問題点について述べる。

最後に、第\ref{chap:conclusion}章で本論文のまとめと結論を述べる。